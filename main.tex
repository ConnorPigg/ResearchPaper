\documentclass{article}
\usepackage[utf8]{inputenc}


\title{Monitoring Parsl Work-flows}
\author{Connor Pigg and Dr. Daniel Katz}
\date{July 2018}

\usepackage{natbib}
\usepackage{graphicx}
\usepackage[colorlinks=true,linkcolor=black,anchorcolor=black,citecolor=black,filecolor=black,menucolor=black,runcolor=black,urlcolor=black]{hyperref}

\begin{document}

\maketitle

\section{Abstract}
As a workflow software, Parsl provides users additional features in addition to the core definition and execution of workflows.
One useful feature that Parsl lacked was convenient workflow monitoring.
This project set out to add workflow monitoring to Parsl as to enhance its feature set.
Simple and comprehensive monitoring of a workflows state and resource usage provides users value by allowing auditing, debugging, and confirmation of workflow execution.
The paper discusses the components of workflows that are monitored by Parsl, the strategy for capturing these components, and the tools used to accumulate and display these components.
Primarily, this project used python libraries to collect task status and resource usage and log these tasks to an Elasticsearch database.
A Kibana dashboard was then created to visualize the collected logs in a real time and interactive UI.
In the end, the newer versions of Parsl will allow users the option to monitor the status and resource usage of their workflows via an Elasticsearch database and Kibana dashboard.

\pagebreak

\section{Introduction}
%should probably add a citation for scientific workflow systems and their demand
Modern scientific practices increasingly incorporate computational elements into their processes.
For computational science, these elements almost entirely comprise any given process.
The consistent increase in the significance and abundance of computational elements drives the development of computational tools.
One such tool for the modern scientist is workflow software.
This software allows the definition and execution of a series of computational elements.
Users benefit from using workflow software by automating their task execution, documenting the steps taken in their process, and empowering themselves with the additional features the software may include.

The motivation for the use and development of such software is clear.
% Workflow software usage is not ubiquitous.
Workflow systems have the potential to bring the full power of the current state of computational resources to the users by optimizing the scheduling and execution of computational workflow tasks.
Nevertheless, workflow software is not always simple and convenient which may be deterring users.
The Parsl project attempts to address these issues.

Parsl is a library for defining a workflow in python \cite{babuji_yadu_2017_853492}.
Offering a workflow system as a library to a common scripting language greatly reduces the cost to users - providing a familiar interface for defining and executing workflows.
Parsl greatly enhances the composition and execution of arbitrary workflows in the following ways.
The tasks that compose the workflow are dynamically scheduled in parallel according to their dependencies.
Parsl workflow definitions are independent of site resources, allowing Parsl to define task execution for many or arbitrary environments.
These basic properties of Parsl empower users to conveniently define and execute tasks that would otherwise create complicated bottlenecks in a modern research workflow.
The motivation for Parsl development is clear, to enable the relatively simple definition and execution of arbitrary workflows such that users can increase productivity.
In order to facilitate the adoption and use of Parsl, additional features are included alongside the core functionality.
These features have included globus file staging, simple configurations, simple installations, multi-site configurations, and others.
% Probably fix this sentence
Monitoring seems to be an important feature for Parsl to also provide its users.

Adding monitoring to any workflow system allows users the ability to audit, debug, and confirm the execution of their workflow.
This specific feature for Parsl will serve to benefit current users and attract new users.
The goal is to provide a monitoring interface that logs task execution and resource usage activity to a simple user interface.
The present solution path for this feature has been to implement logging and visualization using the open source tools Elasticsearch and Kibana.
% add citations for these projects

This paper presents the details of the requirements to monitor a complete workflow and the viability of using Elasticsearch and Kibana in such a solution.
The details of the solution and the development process of implementing workflow monitoring will follow.


\section{Background and Significance}
Work-flow software is used by researchers and projects to automate the execution of and transition between actions that typically occur in any given process.
Parsl empowers users to utilize the parallel capabilities of computers through implicit parallelization of these work-flows.
This speeds up the execution and reduces human errors in the process.
These tools also give users the ability to audit, publish, and debug their work-flows improving their quality and reproducibility.
To further enable these specific uses of work flow tools, this project is adding monitoring to Parsl.

Parsl's decision to include monitoring at such a relatively young age conveys the importance of monitoring to users of such software.

\section{Potential Benefits}
These monitoring capabilities will improve the user's understanding of the behavior of their work flow.
This allows the detection of abnormalities, errors, and performance of their work flow.
This information improves the efficiency at which a user can solve work flow problems and achieve results.

Monitoring should also increase adoption of automated work-flows.
Increased adoption would improve the productivity of these users and empower these users to contribute more to their communities.

\section{Methods}
The work in this paper was performed according to a simple strategy.
First, from the inside Parsl source log information. \textcolor{red}{I don't understand the previous - maybe the grammar needs fixing?}
Second, store these logs to a central location that a user can access.
Finally, present these logs in a dashboard accessible to a user. \textcolor{red}{maybe: present the information in these logs using a dashboard accessible to a user}
The particular source logging implementation uses the Python logging handler CMRESHandler.
An Elasticsearch instance is used for central log storage.
Finally, a Kibana dashboard was chosen for the UI.
The benefit of these options is that they allow simple solution development and provide the core functionality out-of-the-box.
The downside of these options is that they are not trivial to distribute to users and customization can be limited.

Specifically, information about the status and resource usage is generated from the Parsl source code during execution.
This information is then collected on a central Elasticsearch instance \footnote{Since Elasticsearch is a distributed database, information only needs to be sent to a single node and then is propagated by Elasticsearch itself.}
This Elasticsearch instance is then exposed to a Kibana instance.
Kibana hosts the dashboards that present the information to users; they are live and interactive.
Because these instances only require a simple connection, the services may be run from arbitrary systems and only require a user to have access to the Kibana instance in order to monitor the Parsl workflow.

Kibana and Elasticsearch are full tools in their own right and leave plenty of opportunity for this solution to be expanded by the Parsl team or customized by the users running these services.
In order to provide a default that satisfies the majority of users, a generic Kibana dashboard template and Elasticsearch ``schema'' \footnote{Elasticsearch is flexible and does not require a traditional database schema to function. However, to provide smooth interactions with Kibana a ``schema'' in the form of an index pattern is provided by creating dummy entries that follow the prescribed pattern during the setup process.} are provided for use during the initial setup.

The information and visualizations are not unique and could readily be reproduced using online resources and an understanding of any database and visualization tools.
The information generation within Parsl is the component of interest.
Information is primarily collected in two distinct places, by the task scheduler and by the task executioners (workers).
The scope of the task scheduler includes information of overall workflow status and all scheduling events. During each of the scheduling events, useful information is packaged into a log that is sent to Elasticsearch.
This log packaging follows a prescribed pattern in order to smoothly interact with Kibana visualization tools.
On the task executioner, a new process is spawned that monitors the resource usage of the specific running task.
Again, the information is packaged solely for visualization purposes, as Elasticsearch is capable of storing arbitrary and flexible entries.

The solution was developed by iteratively adding logging statements to the code and then crafting visuals for the new information.
This was an efficient way to increase presented information while keeping each addition simple.
% To-Do: this section could be expanded a lot to explain how data is collected (where data is generated (the parsl workers), how it is stored (elasticsearch, set up by the run based on a parameter in config with a default?)), how the collected data is analyzed (Kibana pointed to the elasticsearch instance), how the views in Kibana are set (parameters? Inputs?), what the user actually sees (screenshots), what the user can do to work with the data.


\section{Conclusion}
A monitoring solution has been added to the Parsl workflow software.
This solution uses Elasticsearch and Kibana to store and present recorded workflow information.
The significance of this project is that it clearly outlines a simple and effective way to provide information to users independent of the execution location.
Clear use cases for such techniques are in live monitoring of the execution of workflows that are being executed on arbitrary and potentially distributed locations.
This solution also demonstrates how monitoring may be simplified by separating the components into specialized tools.

The present solution lacks user testing.
Such feedback and experience reports should be collected and incorporated into the UI to increase the usability of monitoring.
Future iterations of monitoring in Parsl will ideally incorporate such feedback.
More customizable solutions for monitoring may be achieved by using the same strategies but with different visualization tools.

In summary, this work has demonstrated that remote monitoring of arbitrary software can be effectively achieved using straightforward tools such as Elasticsearch and Kibana.


\bibliographystyle{plain}
\bibliography{references}
\end{document}
