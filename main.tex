\documentclass{article}
\usepackage[utf8]{inputenc}

\title{\textbf{RIGHT NOW THIS IS A COPY OF THE OUTLINE}
Monitoring Parsl Work Flows}
\author{Connor Pigg and Dr. Daniel Katz}
\date{July 2018}

\usepackage{natbib}
\usepackage{graphicx}

\begin{document}

\maketitle

\section{Introduction}
%should probably add a citation for scientific workflow systems and their demand
Modern scientific practices increasingly incorporate computational elements into their processes.
For computational science, these elements almost entirely comprise any given process.
The consistent increase in the significance and abundance of computational elements drives the development of computational tools.
One such tool for the modern scientist is workflow software.
This software allows the definition and execution of a series of computational elements.
Users benefit from using workflow software by automating their task execution, documenting the steps taken in their process, and empowering themselves with the additional features the software may include.

The motivation for the use and development of such software is clear.
% Workflow software usage is not ubiquitous.
Workflow systems have the potential to bring the full power of the current state of computational resources to the users by optimizing the scheduling and execution of computational workflow tasks.
Nevertheless, workflow software is not always simple and convenient which may be deterring users.
The Parsl project attempts to address these issues.

Parsl is a library for defining a workflow in python \cite{babuji_yadu_2017_853492}.
Offering a workflow system as a library to a common scripting language greatly reduces the cost to users - providing a familiar interface for defining and executing workflows.
Parsl greatly enhances the composition and execution of arbitrary workflows in the following ways.
The tasks that compose the workflow are dynamically scheduled in parallel according to their dependencies.
Parsl workflow definitions are independent of site resources, allowing Parsl to define task execution for many or arbitrary environments.
These basic properties of Parsl empower users to conveniently define and execute tasks that would otherwise create complicated bottlenecks in a modern research workflow.
The motivation for Parsl development is clear, to enable the relatively simple definition and execution of arbitrary workflows such that users can increase productivity.
In order to facilitate the adoption and use of Parsl, additional features are included alongside the core functionality.
These features have included globus file staging, simple configurations, simple installations, multi-site configurations, and others.
% Probably fix this sentence
Monitoring seems to be an important feature for Parsl to also provide its users.

Adding monitoring to any workflow system allows users the ability to audit, debug, and confirm the execution of their workflow.
This specific feature for Parsl will serve to benefit current users and attract new users.
The goal is to provide a monitoring interface that logs task execution and resource usage activity to a simple user interface.
The present solution path for this feature has been to implement logging and visualization using the open source tools Elasticsearch and Kibana.
% add citations for these projects

This paper presents the details of the requirements to monitor a complete workflow and the viability of using Elasticsearch and Kibana in such a solution.
The details of the solution and the development process of implementing workflow monitoring will follow.


\section{Introduction OLD}
Parsl is a library for crafting work flows.\citep{babuji_yadu_2017_853492}
Parsl can be used to implicitly parallelize execution.
The execution graph is derived from the data dependencies between these apps and functions.
The intent is to give users, especially computational researchers, the tools to efficiently create, execute, and interact with the apps and data that make up their work flows.
My contributions to this project will be adding the ability to monitor the execution and performance of a work flow in order to confirm, audit, or debug behavior.

\section{Project Goals}
The purpose of my project is to provide users with the ability to monitor the behavior of their work flows.
The plan is to create a software solution built using the project's existing tools and/or additional libraries.
Ideally, the monitoring should provide an abundance of information about the behavior of execution in a clear visual format.
Specifically, the outline of my research is as follows.
I will familiarize myself with the project and its team to understand the context and resources that my solution should harmonize with.
I will explore available additional resources and tools I find or are recommended such as Kibana, Elasticsearch, and psutil.
Next, I will begin prototyping and iteratively build a solution that incrementally adds features to monitoring.
Finally, I will assess the solution and prepare a finalized version of the work for the completion of the project.

\section{Background and Significance}
Work flow software is used by researchers and projects to automate the transition between actions that typically occur in any given process.
Parsl empowers users to utilize the parallel capabilities of computers through implicit parallelization of these work flows.
This speeds up the process and reduces human errors in the process.
These tools also give users the ability to audit, publish, and debug their work flows improving the quality and reproducability of a process.
To further enable these specific uses of work flow tools, this project will add monitoring to Parsl.

Parsl's decision to include monitoring at such a relatively young age conveys the importance of monitoring to users of such software.

\section{Potential Benefits}
These monitoring capabilities will improve the user's understanding of the behavior of their work flow.
This allows the detection of abnormalities, errors, and performance of their work flow.
This information improves the efficiency at which a user can solve work flow problems and achieve results.

Monitoring should also increase adoption of automated work flows.
Increased adoption would improve the productivity of these users and empower these users to contribute more to their communities.

\section{Conclusion}
Parsl is a useful tool for creating, managing, and soon to be monitoring scientific work flows.
This project will improve Parsl and empower its users as it begins to satisfy the need to observe the behavior of their applications.

I will go through the design process for computational research tools and build a solution for monitoring.
This project will empower the project with monitoring features and me with a greater knowledge of research work flows and the development process.

\bibliographystyle{plain}
\bibliography{references}
\end{document}
